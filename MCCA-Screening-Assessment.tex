% Options for packages loaded elsewhere
\PassOptionsToPackage{unicode}{hyperref}
\PassOptionsToPackage{hyphens}{url}
%
\documentclass[
]{article}
\usepackage{amsmath,amssymb}
\usepackage{iftex}
\ifPDFTeX
  \usepackage[T1]{fontenc}
  \usepackage[utf8]{inputenc}
  \usepackage{textcomp} % provide euro and other symbols
\else % if luatex or xetex
  \usepackage{unicode-math} % this also loads fontspec
  \defaultfontfeatures{Scale=MatchLowercase}
  \defaultfontfeatures[\rmfamily]{Ligatures=TeX,Scale=1}
\fi
\usepackage{lmodern}
\ifPDFTeX\else
  % xetex/luatex font selection
\fi
% Use upquote if available, for straight quotes in verbatim environments
\IfFileExists{upquote.sty}{\usepackage{upquote}}{}
\IfFileExists{microtype.sty}{% use microtype if available
  \usepackage[]{microtype}
  \UseMicrotypeSet[protrusion]{basicmath} % disable protrusion for tt fonts
}{}
\makeatletter
\@ifundefined{KOMAClassName}{% if non-KOMA class
  \IfFileExists{parskip.sty}{%
    \usepackage{parskip}
  }{% else
    \setlength{\parindent}{0pt}
    \setlength{\parskip}{6pt plus 2pt minus 1pt}}
}{% if KOMA class
  \KOMAoptions{parskip=half}}
\makeatother
\usepackage{xcolor}
\usepackage[margin=1in]{geometry}
\usepackage{color}
\usepackage{fancyvrb}
\newcommand{\VerbBar}{|}
\newcommand{\VERB}{\Verb[commandchars=\\\{\}]}
\DefineVerbatimEnvironment{Highlighting}{Verbatim}{commandchars=\\\{\}}
% Add ',fontsize=\small' for more characters per line
\usepackage{framed}
\definecolor{shadecolor}{RGB}{248,248,248}
\newenvironment{Shaded}{\begin{snugshade}}{\end{snugshade}}
\newcommand{\AlertTok}[1]{\textcolor[rgb]{0.94,0.16,0.16}{#1}}
\newcommand{\AnnotationTok}[1]{\textcolor[rgb]{0.56,0.35,0.01}{\textbf{\textit{#1}}}}
\newcommand{\AttributeTok}[1]{\textcolor[rgb]{0.13,0.29,0.53}{#1}}
\newcommand{\BaseNTok}[1]{\textcolor[rgb]{0.00,0.00,0.81}{#1}}
\newcommand{\BuiltInTok}[1]{#1}
\newcommand{\CharTok}[1]{\textcolor[rgb]{0.31,0.60,0.02}{#1}}
\newcommand{\CommentTok}[1]{\textcolor[rgb]{0.56,0.35,0.01}{\textit{#1}}}
\newcommand{\CommentVarTok}[1]{\textcolor[rgb]{0.56,0.35,0.01}{\textbf{\textit{#1}}}}
\newcommand{\ConstantTok}[1]{\textcolor[rgb]{0.56,0.35,0.01}{#1}}
\newcommand{\ControlFlowTok}[1]{\textcolor[rgb]{0.13,0.29,0.53}{\textbf{#1}}}
\newcommand{\DataTypeTok}[1]{\textcolor[rgb]{0.13,0.29,0.53}{#1}}
\newcommand{\DecValTok}[1]{\textcolor[rgb]{0.00,0.00,0.81}{#1}}
\newcommand{\DocumentationTok}[1]{\textcolor[rgb]{0.56,0.35,0.01}{\textbf{\textit{#1}}}}
\newcommand{\ErrorTok}[1]{\textcolor[rgb]{0.64,0.00,0.00}{\textbf{#1}}}
\newcommand{\ExtensionTok}[1]{#1}
\newcommand{\FloatTok}[1]{\textcolor[rgb]{0.00,0.00,0.81}{#1}}
\newcommand{\FunctionTok}[1]{\textcolor[rgb]{0.13,0.29,0.53}{\textbf{#1}}}
\newcommand{\ImportTok}[1]{#1}
\newcommand{\InformationTok}[1]{\textcolor[rgb]{0.56,0.35,0.01}{\textbf{\textit{#1}}}}
\newcommand{\KeywordTok}[1]{\textcolor[rgb]{0.13,0.29,0.53}{\textbf{#1}}}
\newcommand{\NormalTok}[1]{#1}
\newcommand{\OperatorTok}[1]{\textcolor[rgb]{0.81,0.36,0.00}{\textbf{#1}}}
\newcommand{\OtherTok}[1]{\textcolor[rgb]{0.56,0.35,0.01}{#1}}
\newcommand{\PreprocessorTok}[1]{\textcolor[rgb]{0.56,0.35,0.01}{\textit{#1}}}
\newcommand{\RegionMarkerTok}[1]{#1}
\newcommand{\SpecialCharTok}[1]{\textcolor[rgb]{0.81,0.36,0.00}{\textbf{#1}}}
\newcommand{\SpecialStringTok}[1]{\textcolor[rgb]{0.31,0.60,0.02}{#1}}
\newcommand{\StringTok}[1]{\textcolor[rgb]{0.31,0.60,0.02}{#1}}
\newcommand{\VariableTok}[1]{\textcolor[rgb]{0.00,0.00,0.00}{#1}}
\newcommand{\VerbatimStringTok}[1]{\textcolor[rgb]{0.31,0.60,0.02}{#1}}
\newcommand{\WarningTok}[1]{\textcolor[rgb]{0.56,0.35,0.01}{\textbf{\textit{#1}}}}
\usepackage{graphicx}
\makeatletter
\def\maxwidth{\ifdim\Gin@nat@width>\linewidth\linewidth\else\Gin@nat@width\fi}
\def\maxheight{\ifdim\Gin@nat@height>\textheight\textheight\else\Gin@nat@height\fi}
\makeatother
% Scale images if necessary, so that they will not overflow the page
% margins by default, and it is still possible to overwrite the defaults
% using explicit options in \includegraphics[width, height, ...]{}
\setkeys{Gin}{width=\maxwidth,height=\maxheight,keepaspectratio}
% Set default figure placement to htbp
\makeatletter
\def\fps@figure{htbp}
\makeatother
\setlength{\emergencystretch}{3em} % prevent overfull lines
\providecommand{\tightlist}{%
  \setlength{\itemsep}{0pt}\setlength{\parskip}{0pt}}
\setcounter{secnumdepth}{-\maxdimen} % remove section numbering
\ifLuaTeX
  \usepackage{selnolig}  % disable illegal ligatures
\fi
\usepackage{bookmark}
\IfFileExists{xurl.sty}{\usepackage{xurl}}{} % add URL line breaks if available
\urlstyle{same}
\hypersetup{
  pdftitle={MCCA Screening Assessment},
  pdfauthor={Erika Xu},
  hidelinks,
  pdfcreator={LaTeX via pandoc}}

\title{MCCA Screening Assessment}
\author{Erika Xu}
\date{Feb 17, 2025}

\begin{document}
\maketitle

\begin{Shaded}
\begin{Highlighting}[]
\FunctionTok{library}\NormalTok{(tidyr)}
\FunctionTok{library}\NormalTok{(dplyr)}
\end{Highlighting}
\end{Shaded}

\begin{verbatim}
## 
## Attaching package: 'dplyr'
\end{verbatim}

\begin{verbatim}
## The following objects are masked from 'package:stats':
## 
##     filter, lag
\end{verbatim}

\begin{verbatim}
## The following objects are masked from 'package:base':
## 
##     intersect, setdiff, setequal, union
\end{verbatim}

\begin{Shaded}
\begin{Highlighting}[]
\FunctionTok{library}\NormalTok{(ggplot2) }
\FunctionTok{library}\NormalTok{(colorblindr)}
\end{Highlighting}
\end{Shaded}

\begin{verbatim}
## Loading required package: colorspace
\end{verbatim}

The unit of observation for the combined dataset is the firm. Some
initial observations I find interesting within the dataset is that there
are only five variables that provide detail about each firm.
Binary\_has\_dei\_program indicates weather or no a firm has a DEI
program. These is count\_attorneys, represents the total number of
attorneys at the given firm, count\_female, the total number of females,
and count\_minority, the total number of minorities at the given firm.
And mcca\_score seems to indicate how a firm scored against MCCA's
Scorecard.

Also, there are no given dates, and there are duplicates for each firm
entry. Therefore, I assume that this dataset represent a snapshot of a
firm's DEI positioning at a single point in time, rather than reflecting
a firm's DEI progress over time.

With this dataset, I would assume each attribute/column about each firm
will give some information about their MCCA score. I am curious about
why we are looking at these specific attributes in this dataset or if
other variables can be more effective in illustrating charts/ graphs
about a firm's MCCA score. Additionally, I am curious about what is
categorized under ``county\_minority'' since no prior details are
provided about these variables.

\begin{Shaded}
\begin{Highlighting}[]
\NormalTok{attributes }\OtherTok{\textless{}{-}} \FunctionTok{read.csv}\NormalTok{(}\StringTok{"data/dat\_attributes.csv"}\NormalTok{)}
\NormalTok{score }\OtherTok{\textless{}{-}} \FunctionTok{read.csv}\NormalTok{(}\StringTok{"data/dat\_score.csv"}\NormalTok{)}

\FunctionTok{nrow}\NormalTok{ (attributes)}
\end{Highlighting}
\end{Shaded}

\begin{verbatim}
## [1] 820
\end{verbatim}

\begin{Shaded}
\begin{Highlighting}[]
\FunctionTok{colnames}\NormalTok{ (attributes)}
\end{Highlighting}
\end{Shaded}

\begin{verbatim}
## [1] "count_attorneys"        "binary_has_dei_program" "firm_name"             
## [4] "count_female"           "count_minority"
\end{verbatim}

\begin{Shaded}
\begin{Highlighting}[]
\FunctionTok{nrow}\NormalTok{ (score)}
\end{Highlighting}
\end{Shaded}

\begin{verbatim}
## [1] 820
\end{verbatim}

\begin{Shaded}
\begin{Highlighting}[]
\FunctionTok{colnames}\NormalTok{ (score)}
\end{Highlighting}
\end{Shaded}

\begin{verbatim}
## [1] "firm_name"  "mcca_score"
\end{verbatim}

\begin{Shaded}
\begin{Highlighting}[]
\NormalTok{combined }\OtherTok{\textless{}{-}} \FunctionTok{full\_join}\NormalTok{(attributes, score, }\AttributeTok{by =} \StringTok{"firm\_name"}\NormalTok{) }\SpecialCharTok{|\textgreater{}}
  \FunctionTok{select}\NormalTok{(firm\_name, }\FunctionTok{everything}\NormalTok{()) }
   
  
\NormalTok{combined }\OtherTok{\textless{}{-}}\NormalTok{ combined }\SpecialCharTok{|\textgreater{}}
  \FunctionTok{relocate}\NormalTok{(binary\_has\_dei\_program, }\AttributeTok{.after =}\NormalTok{ firm\_name)}
\end{Highlighting}
\end{Shaded}

From the combined dataset, I would like to know whether or not having a
DEI program (binary\_has\_dei\_program) impacts a firm's score on the
MCCA scorecard (mcca\_score). In this case, having a score greater or
equal to 0.60 will be passing.

Around 169 (\textasciitilde20\%) of the total firms have a DEI program
and 651 (\textasciitilde80\%) do not have a DEI program. For firms with
a DEI program, 68 (40\%) have a passing MCCA score. For firms that do
not have DEI programs, 254 (39\%) have a passing MCCA score. This
suggests that having a DEI program does not necessarily correlate
strongly to a passing MCCA score since the passing rates between firms
with DEI programs and those without are very small.

Next, I would like to further explore how variables, like firm size
(count\_attorneys) and diversity metrics (count\_female and
count\_minority) impact the MCCA score. Maybe there will be some
underlying patterns there.

\begin{Shaded}
\begin{Highlighting}[]
\CommentTok{\#DEI programs and mcca\_score }
\NormalTok{combined }\SpecialCharTok{|\textgreater{}}
  \FunctionTok{count}\NormalTok{(binary\_has\_dei\_program)}
\end{Highlighting}
\end{Shaded}

\begin{verbatim}
##   binary_has_dei_program   n
## 1                  FALSE 651
## 2                   TRUE 169
\end{verbatim}

\begin{Shaded}
\begin{Highlighting}[]
\FunctionTok{sum}\NormalTok{(combined}\SpecialCharTok{$}\NormalTok{binary\_has\_dei\_program }\SpecialCharTok{==} \ConstantTok{TRUE} \SpecialCharTok{\&}\NormalTok{ combined}\SpecialCharTok{$}\NormalTok{mcca\_score }\SpecialCharTok{\textgreater{}=} \FloatTok{0.60}\NormalTok{)}
\end{Highlighting}
\end{Shaded}

\begin{verbatim}
## [1] 68
\end{verbatim}

\begin{Shaded}
\begin{Highlighting}[]
\DecValTok{68}\SpecialCharTok{/}\DecValTok{169}
\end{Highlighting}
\end{Shaded}

\begin{verbatim}
## [1] 0.4023669
\end{verbatim}

\begin{Shaded}
\begin{Highlighting}[]
\FunctionTok{sum}\NormalTok{(combined}\SpecialCharTok{$}\NormalTok{binary\_has\_dei\_program }\SpecialCharTok{==} \ConstantTok{FALSE} \SpecialCharTok{\&}\NormalTok{ combined}\SpecialCharTok{$}\NormalTok{mcca\_score }\SpecialCharTok{\textgreater{}=} \FloatTok{0.60}\NormalTok{)}
\end{Highlighting}
\end{Shaded}

\begin{verbatim}
## [1] 254
\end{verbatim}

\begin{Shaded}
\begin{Highlighting}[]
\DecValTok{254}\SpecialCharTok{/}\DecValTok{651}
\end{Highlighting}
\end{Shaded}

\begin{verbatim}
## [1] 0.390169
\end{verbatim}

\begin{Shaded}
\begin{Highlighting}[]
\FunctionTok{sum}\NormalTok{(combined}\SpecialCharTok{$}\NormalTok{mcca\_score }\SpecialCharTok{\textgreater{}} \FloatTok{0.600}\NormalTok{)}
\end{Highlighting}
\end{Shaded}

\begin{verbatim}
## [1] 322
\end{verbatim}

There seems to be no correlations between MCCA scores (mcca\_score) and
firm size. Whether or not a firm has a DEI program also does not appear
to correlate with the firm size.

\begin{Shaded}
\begin{Highlighting}[]
\CommentTok{\#group firms by size}
\FunctionTok{summary}\NormalTok{(combined}\SpecialCharTok{$}\NormalTok{count\_attorneys)}
\end{Highlighting}
\end{Shaded}

\begin{verbatim}
##    Min. 1st Qu.  Median    Mean 3rd Qu.    Max. 
##     1.0   264.8   502.5   502.8   747.0  1000.0
\end{verbatim}

\begin{Shaded}
\begin{Highlighting}[]
\NormalTok{combined }\OtherTok{\textless{}{-}}\NormalTok{ combined }\SpecialCharTok{|\textgreater{}}
  \FunctionTok{mutate}\NormalTok{(}\AttributeTok{firm\_size =} \FunctionTok{cut}\NormalTok{(count\_attorneys, }
                         \AttributeTok{breaks =} \FunctionTok{c}\NormalTok{(}\DecValTok{1}\NormalTok{, }\DecValTok{264}\NormalTok{, }\DecValTok{502}\NormalTok{, }\DecValTok{747}\NormalTok{, }\DecValTok{1000}\NormalTok{), }
                         \AttributeTok{labels =} \FunctionTok{c}\NormalTok{(}\StringTok{"Very Small"}\NormalTok{, }\StringTok{"Small"}\NormalTok{, }\StringTok{"Medium"}\NormalTok{, }\StringTok{"Large"}\NormalTok{)))}
\FunctionTok{summary}\NormalTok{(combined}\SpecialCharTok{$}\NormalTok{firm\_size)}
\end{Highlighting}
\end{Shaded}

\begin{verbatim}
## Very Small      Small     Medium      Large       NA's 
##        204        205        206        204          1
\end{verbatim}

\begin{Shaded}
\begin{Highlighting}[]
\CommentTok{\#MCCA Score vs Firm Size}
\FunctionTok{ggplot}\NormalTok{(combined, }\FunctionTok{aes}\NormalTok{(}\AttributeTok{x =}\NormalTok{ count\_attorneys, }\AttributeTok{y =}\NormalTok{ mcca\_score, }\AttributeTok{color =}\NormalTok{ binary\_has\_dei\_program)) }\SpecialCharTok{+}
  \FunctionTok{geom\_point}\NormalTok{() }\SpecialCharTok{+}
  \FunctionTok{facet\_wrap}\NormalTok{(}\SpecialCharTok{\textasciitilde{}}\NormalTok{ firm\_size, }\AttributeTok{scales =} \StringTok{"free"}\NormalTok{) }\SpecialCharTok{+}
  \FunctionTok{labs}\NormalTok{(}\AttributeTok{x =} \StringTok{"Number of Attorneys (Firm Size)"}\NormalTok{, }
       \AttributeTok{y =} \StringTok{"MCCA Score"}\NormalTok{, }
       \AttributeTok{title =} \StringTok{"MCCA Score vs Firm Size"}\NormalTok{) }\SpecialCharTok{+}
  \FunctionTok{theme\_minimal}\NormalTok{() }\SpecialCharTok{+}
  \FunctionTok{scale\_fill\_OkabeIto}\NormalTok{()}
\end{Highlighting}
\end{Shaded}

\includegraphics{MCCA-Screening-Assessment_files/figure-latex/unnamed-chunk-5-1.pdf}

After reviewing the scatter plots for potential correlations between
MCCA scores (mcca\_score) and the diversity metrics in the dataset
(count\_female and count\_minority), there are no correlations. Whether
or not a firm has a DEI program also does not appear to correlate with
the percentage of female or minority attorneys. Although, it is worth
noting that most firms have 30\% or more female attorneys.

\begin{Shaded}
\begin{Highlighting}[]
\CommentTok{\#female\_percent}
\NormalTok{combined }\OtherTok{\textless{}{-}}\NormalTok{ combined }\SpecialCharTok{|\textgreater{}} 
  \FunctionTok{mutate}\NormalTok{(}\AttributeTok{female\_percent =}\NormalTok{ count\_female }\SpecialCharTok{/}\NormalTok{ count\_attorneys)}

\CommentTok{\#minority\_percent}
\NormalTok{combined }\OtherTok{\textless{}{-}}\NormalTok{ combined }\SpecialCharTok{|\textgreater{}} 
  \FunctionTok{mutate}\NormalTok{(}\AttributeTok{minority\_percent =}\NormalTok{ count\_minority }\SpecialCharTok{/}\NormalTok{ count\_attorneys)}

\CommentTok{\#MCCA Score vs Female Ratio}
\FunctionTok{ggplot}\NormalTok{(combined, }\FunctionTok{aes}\NormalTok{(}\AttributeTok{x =}\NormalTok{ female\_percent, }\AttributeTok{y =}\NormalTok{ mcca\_score)) }\SpecialCharTok{+}
  \FunctionTok{geom\_point}\NormalTok{(}\FunctionTok{aes}\NormalTok{(}\AttributeTok{color =} \FunctionTok{factor}\NormalTok{(binary\_has\_dei\_program))) }\SpecialCharTok{+}
  \FunctionTok{labs}\NormalTok{(}\AttributeTok{x =} \StringTok{"Female Percentage"}\NormalTok{, }
       \AttributeTok{y =} \StringTok{"MCCA Score"}\NormalTok{, }
       \AttributeTok{title =} \StringTok{"MCCA Score vs Female Ratio"}\NormalTok{) }\SpecialCharTok{+}
  \FunctionTok{scale\_color\_discrete}\NormalTok{(}\AttributeTok{name =} \StringTok{"Has DEI Program"}\NormalTok{, }\AttributeTok{labels =} \FunctionTok{c}\NormalTok{(}\StringTok{"No"}\NormalTok{, }\StringTok{"Yes"}\NormalTok{)) }\SpecialCharTok{+}
  \FunctionTok{scale\_fill\_OkabeIto}\NormalTok{()}
\end{Highlighting}
\end{Shaded}

\includegraphics{MCCA-Screening-Assessment_files/figure-latex/unnamed-chunk-6-1.pdf}

\begin{Shaded}
\begin{Highlighting}[]
\CommentTok{\#MCCA Score vs Minority Ratio}
\FunctionTok{ggplot}\NormalTok{(combined, }\FunctionTok{aes}\NormalTok{(}\AttributeTok{x =}\NormalTok{ minority\_percent, }\AttributeTok{y =}\NormalTok{ mcca\_score)) }\SpecialCharTok{+}
  \FunctionTok{geom\_point}\NormalTok{(}\FunctionTok{aes}\NormalTok{(}\AttributeTok{color =} \FunctionTok{factor}\NormalTok{(binary\_has\_dei\_program))) }\SpecialCharTok{+}
  \FunctionTok{labs}\NormalTok{(}\AttributeTok{x =} \StringTok{"Minority Percentage"}\NormalTok{, }
       \AttributeTok{y =} \StringTok{"MCCA Score"}\NormalTok{, }
       \AttributeTok{title =} \StringTok{"MCCA Score vs Minority Ratio"}\NormalTok{) }\SpecialCharTok{+}
  \FunctionTok{scale\_color\_discrete}\NormalTok{(}\AttributeTok{name =} \StringTok{"Has DEI Program"}\NormalTok{, }\AttributeTok{labels =} \FunctionTok{c}\NormalTok{(}\StringTok{"No"}\NormalTok{, }\StringTok{"Yes"}\NormalTok{)) }\SpecialCharTok{+}
  \FunctionTok{scale\_fill\_OkabeIto}\NormalTok{()}
\end{Highlighting}
\end{Shaded}

\includegraphics{MCCA-Screening-Assessment_files/figure-latex/unnamed-chunk-6-2.pdf}

So, from what I've gathered, a firm's MCCA score has no correlation to
whether or not it has a DEI program, firm size, female ratio, or
minority ratio. Now, I have to ask: What does the MCCA score represent
and how is it scored? Are there specific criteria or metrics that
influence a firm's score that isn't included in this dataset?

\begin{Shaded}
\begin{Highlighting}[]
\CommentTok{\#side quests}
\NormalTok{female }\OtherTok{\textless{}{-}}\NormalTok{ combined }\SpecialCharTok{|\textgreater{}}
  \FunctionTok{filter}\NormalTok{(female\_percent }\SpecialCharTok{==} \DecValTok{0}\NormalTok{)}

\NormalTok{high\_scoring\_firms }\OtherTok{\textless{}{-}}\NormalTok{ combined }\SpecialCharTok{|\textgreater{}}
  \FunctionTok{filter}\NormalTok{(mcca\_score }\SpecialCharTok{\textgreater{}} \FloatTok{0.80}\NormalTok{)}

\NormalTok{low\_scoring\_firms }\OtherTok{\textless{}{-}}\NormalTok{ combined }\SpecialCharTok{|\textgreater{}}
  \FunctionTok{filter}\NormalTok{(mcca\_score }\SpecialCharTok{\textless{}} \FloatTok{0.20}\NormalTok{)}
\end{Highlighting}
\end{Shaded}


\end{document}
